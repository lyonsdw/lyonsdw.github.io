<section xml:id="sets_counting_section"  xmlns:xi="http://www.w3.org/2001/XInclude">
<title>
Sets and Counting
</title>

<subsection><title>Multiplication Principle(s)</title>

  <p>Given a finite set <m>S</m>, recall that <m>|S|</m> denotes the
    number of elements in <m>S</m>, also called the <em>size</em> of the
    set <m>S</m>.
    Given sets <m>U,V</m>, recall that <m>U\times V</m>
    (called the <em>Cartesian product</em>) is 
    set of all ordered pairs
    <me>U\times V = \{(u,v)\colon u\in U,v\in V\}.</me>
  </p>
  
  <proposition><title>The multiplication principle</title>
    <p>Let <m>A,B</m> be finite sets. We have
      <men>|A\times B|=|A|\;|B|.</men>
    </p>
  </proposition>

  <exercise><statement><p>Give an argument to explain why the
  multiplication rule is true.</p>
    </statement>
  </exercise>

  <p>Recall that two sets are <em>disjoint</em> if their intersection is
    empty. We say that a collection <m>\mathcal{A}</m>
    of sets is <term>pairwise
      disjoint</term> if <m>U,V</m> are disjoint for every pair of
    sets <m>U,V</m> in <m>\mathcal{A}</m>.
    Recall that a <em>partition</em> of a set <m>S</m> is a collection
    of nonempty, pairwise disjoint subsets of <m>S</m> whose union
    is <m>S</m>.
  </p>
  
  <proposition>
    <title>Counting by partitions</title>
    <statement>
      <p>Let <m>\mathcal{A}</m> be a partition of a finite
	set <m>S</m>. Then we have
	<men>|S|=\sum_{U\in \mathcal{A}}|U|. </men>
      </p>
    </statement>
  </proposition>

  <corollary xml:id="partmultprin">
    <title>Partition version of the multiplication principle</title>
    <statement><p>
	Suppose a finite set <m>S</m> is partitioned into <m>k</m>
	subsets, each of which has the same size <m>\ell</m>. Then
	<me>|S|=k\ell.</me>
      </p>
    </statement>
  </corollary>

  <p><xref ref="partmultprin"/> provides a way to count the elements in
    a set using a sequence of <em>decisions</em>, or <em>choices</em>. If a set <m>S</m>
    is partitioned into subsets of equal size, we can choose an element
    of <m>S</m> as follows.
    <ol>
      <li>First, choose one of the partition subsets.</li>
      <li>Second, choose an element from that subset.</li>      
    </ol>
    Since there are <m>k</m> ways to make the first choice, and then,
    having made the first choice, there
    are <m>\ell</m> ways to make the second choice, then we conclude
    that <m>|S|=k\ell</m>.
    This thinking generalizes to longer choice sequences, as follows.
  </p>

  <proposition>
    <title>Sequence-of-decisions version of the multiplication
      principle</title>
<statement>    <p>Suppose that every element of a finite set <m>S</m> is uniquely
      specified by a sequence of <m>r</m>
      choices, such that there are <m>m_1</m>
      ways to make the first choice; then, having made the first choice,
      there are <m>m_2</m> ways to make the second choice; and so on;
      finally, having made the <m>(r-1)</m>st choice, there
      are <m>m_r</m> ways to make the <m>r</m>th choice. Then we have
      <men>|S|=m_1m_2\cdots m_r.</men>
       </p>
  </statement>
  </proposition>

  <p>As an application, we count the number of orderings of a finite
    set. Recall that a bijection is a function that is one-to-one and
    onto. An <em>ordering</em> or <em>permutation</em>
    of a finite set <m>A</m> of
    size <m>|A|=n</m> is given by a
    bijection <m>f\colon \{1,2,\ldots,n\}\to A</m>. We say the first
    element of <m>A</m> is <m>f(1)</m>, the second element
    is <m>f(2)</m>, and so on. We can specify an ordering of <m>A</m> by
    making the following sequence of choices.
    <ol>
      <li>First, choose a value for <m>f(1)</m>. There are <m>n</m> ways
      to make this choice.
      </li>
      <li>Second, from the <m>n-1</m> elements of <m>A</m> that were not
	chosen in step 1, choose a value for <m>f(2)</m>. There
	are <m>n-1</m> ways to make this choice.
      </li>
	<li>And so on. </li>      
    </ol>
    From this sequence of <m>n</m> choices, we conclude that the number
    of orderings of <m>A</m> is <m>n!= n(n-1)(n-2)\ldots 3\cdot 2\cdot
    1</m>. We record this result as a proposition.
  </p>

  <proposition>
    <title>Counting orderings of a finite set</title>
    <statement><p>Let <m>A</m> be a finite set of size <m>|A|=n</m>. The number of orderings
	(or permutations) of the elements of <m>A</m> is <m>n!</m>.
      </p>
    </statement>
  </proposition>
  
  <!--
  <definition>
  <statement><p>Let <m>S,T</m> be sets. We write <m>T^S</m> to denote the set of
    all functions from <m>S</m> to <m>T</m>.
      <me>T^S = \{f\colon S\to T\}</me></p>
  </statement>    
  </definition>

  <proposition xml:id="countfunctionsprop">
    <p>Let <m>S,T</m> be finite sets. We have
            <men>|T^S| = |T|^{|S|}.</men>
    </p>
  </proposition>

  <exercise>
    <statement><p>Use the multiplication principle to prove
    <xref ref="countfunctionsprop"/>. </p>
    </statement>
  </exercise>
  -->
</subsection>

<subsection>
  <title>Sequences</title>
  
  <p>Let <m>S</m> be a set. A function <m>f\colon \{1,2,\ldots,n\}\to
      S</m> is called a <em>sequence</em> in <m>S</m> of
      length <m>n</m>. A sequence is denoted by an ordered
      list <m>(s_1,s_2,\ldots,s_n)</m> of elements in <m>S</m>,
      where <m>s_k=f(k)</m> for <m>1\leq k\leq n</m>.  The set of all
      possible sequences in <m>S</m> of length <m>n</m> is denoted
    <m>S^n</m>, and is also called the <m>n</m>-fold <em>Cartesian
      product</em> of <m>S</m> with itself.
    <men>S^n = \underbrace{S\times S \times \cdots \times S}_{n \text{
      factors}}=\{(s_1,s_2,\ldots,s_n)\colon s_k\in S\text{ for } 1\leq
      k\leq n\}</men>
    Elements of <m>S^n</m> are also
    called <em>ordered <m>n</m>-tuples</em>. It is convenient to have a
    name for the subset of sequences <m>f\colon \{1,2,\ldots,n\}\to
      S</m> that are one-to-one, which is the same as the set of 
    <m>n</m>-tuples <m>(s_1,s_2,\ldots,s_n)</m> that have no repeated
    values, that is, for
    which <m>s_i\neq s_j</m> for <m>i\neq j</m>. In these notes, we will
    write <m>S^{n\ast}</m> to denote the set of sequences that are
    one-to-one.<fn>The notation <m>S^{n\ast}</m> is not standard. In
      these notes, we do <em>not</em> use the standard notation is <m>P_n(S)</m> (where <m>P</m> stands
      for <em>permutations</em>) in order to avoid confusion with
      probability measures, which are also denoted using <m>P</m>.
    </fn>
    <men>S^{n\ast} =\{(s_1,s_2,\ldots,s_n)\in S^n\colon s_i\neq
      s_j\text{ for }i\neq j\}</men>
  </p>

    <exercise><p>
      Let <m>\Omega=\{x,y\}</m>.
      <ol>
	<li>Write out the elements of the set <m>\Omega^3</m>.
	</li>
	<li>Write out the elements of the set <m>\Omega^{2\ast}</m>.
	</li>
	<li>Write out the elements of the set <m>\mathcal{P}(\Omega)</m>.
	</li>
	<li>Write out the elements of the set <m>\mathcal{P}(\Omega^{2\ast})</m>.
	</li>	
      </ol>
    </p>
  </exercise>

    <proposition>
      <title>Counting sequences</title>
      <statement>
	<p>Let <m>S</m> be a finite set, with <m>|S|=N</m>. We have the following.
	  <mdn>
	    <mrow>|S^n| = N^n</mrow>
	    <mrow xml:id="permNnformula">|S^{n\ast}| = N(N-1)(N-2)\cdots (N-n+1)</mrow>	    
	  </mdn>
	</p>
      </statement>
    </proposition>
    
    <exercise><statement><p>give arguments why, use decision sequence
	  version of multiplication principle for <m>S^{n\ast}</m>
	</p>
      </statement>
    </exercise>


    <p>As an application, we count the number of subsets of a fixed size
      that come from a given finite set. Let <m>S</m> be a finite set of size <m>|S|=N</m>, let <m>n</m>
      be an integer in the range <m>0\leq n\leq N</m>,
      and let <m>\mathcal{A}</m> be the set of
      all subsets of <m>S</m> of size <m>n</m>. An element 
      of <m>S^{n\ast}</m> is specified by the following sequence of
      decisions.
      <ol>
	<li>First, choose an element <m>A</m>
	  of <m>\mathcal{A}</m>.
	</li>
	<li>Second, choose an ordering of <m>A</m>.
	</li>	
      </ol>
The sequence-of-choices version of the multiplication principle applies,
      and we have
      <me>\#(\text{subsets of }S\text{ of size }n) \cdot \#(\text{orderings of an }n\text{ element set}) = |S^{n*}|.</me>
      Putting this together with results above, we have
            <men xml:id="cNnformula">\#(\text{subsets of }S\text{ of size } n)
            =\frac{N(N-1)\cdots (N-n+1)}{n!}=\frac{N!}{n! \;
              (N-n)!}.</men>
	    The number given in <xref ref="permNnformula"/> is called the <term>number
              of permutations of <m>N</m> items chosen <m>n</m> at a time
	    </term>. The number given in <xref ref="cNnformula"/> the
	    <term>number of combinations of <m>N</m> items
	    chosen <m>n</m> at a time</term>. The usual mathematical
	    notation for the latter is <m>{N \choose n}</m>.
	    
  </p>

    <p>exercise section: connect N choose n to binomial expansion, give
    terminology binomial coefficient</p>
    
</subsection>



</section>
