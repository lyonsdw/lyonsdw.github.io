% PED 137 Taekwondo I Fall 2006 Syllabus

\documentclass[11pt]{article}
%\documentclass{article}

% this is needed so that pdflatex does not confuse the printer with a
% demand for A4 paper
\usepackage[letterpaper]{geometry}

%%%%%%%%%%%%%%%     preamble    %%%%%%%%%%%%%%%%%%%

%The following lines give the page setup
  %this sets paragraph indentation to zero
  %  \setlength{\parindent}{0in}
  %this sets the width of the text
    \setlength{\textwidth}{5.5in}
  %this sets the left margin; zero is one inch from left edge
    \setlength{\oddsidemargin}{0.5in}
  %this sets text height
    \setlength{\textheight}{9.5in}
  %this sets top margin; zero is one inch from top edge
    \setlength{\topmargin}{0in}
%\addtolength{\topmargin}{-.2in}
  %this produces 1, 1.5, double, etc spaced lines
%   \renewcommand{\baselinestretch}{0.9} 
  %headings
    \pagestyle{myheadings}
     \markright{PED 137 Taekwondo Fall 2006 Syllabus}

\setlength{\parindent}{0in}
\setlength{\parskip}{1ex}

% invisible text ``spacer''
\newcommand{\spacer}{\rule[0cm]{0cm}{0cm}}

% load picture importing package (uncomment usepackage command below)
% for figures created using xfig with 2 part export
%  (pstex and pstex_t files)
% include the figure in this doc with the command
% \input{filename.pstex_t}
% at the point where the figure should appear
%\usepackage{epsfig}

%%%%%%%%%%%%%%% end of preamble %%%%%%%%%%%%%%%%%%%

\begin{document}

\thispagestyle{empty}

\vspace*{-.5in}
  \begin{center}
    {\Large PED 137 {\bf Taekwondo}  Fall 2006}\\{\Large Course Information}
  \end{center}

\vspace*{-3ex}
\subsection*{Instructor}

\vspace*{-1ex}
\begin{tabular}
  {ll}
Instructor: &    Master David Lyons\\
Office:      &   Lynch 283G\\
email:       &   lyons@lvc.edu\\
phone:       &   867-6081\\
web page:    &   http://stuorgs.lvc.edu/tkdclub 
\end{tabular}

\vspace*{-3ex}
\subsection*{Course Description}
\vspace*{-1ex}

 This course is an introduction to basic techniques of the Korean martial art taekwondo.

\vspace*{-3ex}
\subsection*{Course Fee}
\vspace*{-1ex}

There is a participation fee of \$20 for this course.

\vspace*{-3ex}
\subsection*{To Earn Credit}
\vspace*{-1ex}

To earn PE credit for this course, you must attend 12 classes and pay the \$20 participation fee.

You are not required to become a member of the Taekwondo Club, to
purchase a uniform, or participate in belt promotion tests. However, you
are invited to do these things if you wish. 

\vspace*{-3ex}
\subsection*{What to Wear}
\vspace*{-1ex}

Wear comfortable workout clothes that do not restrict your
movement. Please leave watches and jewelery in your dorm room or in a
gym bag.


\vspace*{-3ex}
\subsection*{Etiquette}
\vspace*{-1ex}

 Taekwondo is a martial art, and as such, has a code of etiquette for
 behavior during training. Taekwondo is based on mutual respect among
 fellow students and between teacher and student. In particular,
 students are expected to follow the teacher's instructions, and to ask
 permission when this is not possible. We do this not merely to follow
 tradition, but also for safety.

\vspace*{-3ex}
\subsection*{Email}
\vspace*{-1ex}

Occasional course announcements will be made to your LVC email account.



\end{document}
